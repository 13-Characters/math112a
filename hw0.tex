\documentclass{ben}
\title{MATH 112A Homework 0}
\author{Benjamin Tong}
\date{October 3, 2025}

\begin{document}
    \maketitle
    \section{Problem 1}
    Which is the following operators are linear? Provide reasons.
    \begin{enumerate}[label=(\textbf{\arabic*)}]
        \item $\mathscr{L}u = u_x + e^x u_y$
        \item $\mathscr{L}u = u_x + |u_y|$
        \item $\mathscr{L}u = u_x + u_y + 1$
        \item $\mathscr{L}u = u_t + u_{xxx}$
    \end{enumerate}
    \begin{solution}
        There are two requirements that need to be fufilled in order for $\mathscr{L}$ to be linear.
        The first requirement is that $\mathscr{L}(u + v) = \mathscr{L}u + \mathscr{L}v$, and that
        $\mathscr{L}(cu) = c\mathscr{L}u$.

        According to this defintiion, this means that only (1) and (4) are linear.

        To show that (1) is linear, we have to show that both
        $\mathscr{L}(u + v) = \mathscr{L}u + \mathscr{L}v$ and that
        $\mathscr{L}(cu) = c\mathscr{L}u$. We find that
        \begin{align*}
           \mathscr{L}(u + v) &= (u + v)_x + e^x(u + v)_y\\
           &= u_x + v_x + e^x(u_y + v_y)\\
           &= u_x + e^x u_y + v_x + e^x v_y\\
           &= \mathscr{L} u + \mathscr{L} v
        \end{align*}
        and
        \begin{align*}
            \mathscr{L}(cu) &= (cu)_x + e^x (cu)_y\\
            &= c(u_x) + ce^x (u_y)\\
            &= c(u_x + e^x u_y)\\
            &= c \mathscr{L} u
        \end{align*}
        
        To show that (4) is linear, we have to show that both
        $\mathscr{L}(u + v) = \mathscr{L}u + \mathscr{L}v$ and that
        $\mathscr{L}(cu) = c\mathscr{L}u$. We find that
        \begin{align*}
            \mathscr{L}(u + v) &= (u + v)_t + (u + v)_{xxx}\\
            &= u_t + v_t + u_{xxx} + v_{xxx}\\
            &= (u_t + u_{xxx}) + (v_t + v_{xxx})\\
            &= \mathscr{L}u + \mathscr{L}v
        \end{align*}
        and
        \begin{align*}
            \mathscr{L}(cu) &= (cu)_t + (cu)_{xxx}\\
            &= c(u_t) + c(u_{xxx})\\
            &= c(u_t + u_{xxx})\\
            &= c\mathscr{L}u
        \end{align*}
        Now, (2) and (4) are not linear since they do not meet the two criteria mentioned earlier.
    \end{solution}
    \section{Problem 2}
    For each of the following equations, state the order and whether it is
    nonlinear, linear inhomogeneous, or linear homogeneous.
    \begin{enumerate}[label=(\textbf{\arabic*)}]
        \item $y u_x + x u_y = 0$.
        \item $u_t - u_{xx} + 1 = 0$.
        \item $u_t + u_{xxt} + u u_x = 0$.
        \item $u_{tt} - u(u_x)^3 = 0$.
        \item $u_{tt} - u_{xx} + x^2 = 0$.
    \end{enumerate}
    \begin{multipart}
        \item [\textbf{(1)}] This is a linear homogeneous equation with order 1.
        \item [\textbf{(2)}] This is a linear inhomogeneous equation with order 2.
        \item [\textbf{(3)}] This is a nonlinear equation with order 3.
        \item [\textbf{(4)}] This is a nonlinear equation with order 2.
        \item [\textbf{(5)}] This is a linear inhomogeneous equation with order 2.
    \end{multipart}
    \section{Problem 3}
    Show that the difference of two solutions of an inhomogeneous linear equation
    $\mathscr{L}u = g$ with the same $g$ is a solution of the homogeneous equation
    $\mathscr{L}u = 0$.
    \begin{solution}
        Let $\mathscr{L}$ be a linear operator and let $a$ and $b$ be functions that satisfy
        $\mathscr{L}a = g$ and $\mathscr{L}b = g$. Then, $\mathscr{L}a - \mathscr{L}b = 0$.
        Since $\mathscr{L}$ is a linear operator this means that $\mathscr{L}(a - b) = 0$.
        Therefore, $a - b$ is a solution to the homogeneous equation $\mathscr{L}u = 0$.
    \end{solution}
    \section{Problem 4}
    Verify that $u(x, y) = f(x)g(y)$ is a solution of the PDE
    \[
        uu_{xy} = u_x u_y
    \]
    for all pairs of (differentiable) functions $f$ and $g$ of one variable.
    \begin{solution}
        If $u(x, y) = f(x)g(y)$ for some functions $f$ and $g$, then
        \begin{align*}
            u_x &= f'(x)g(y)\\
            u_y &= f(x)g'(y)\\
            u_{xy} &= f'(x)g'(y)
        \end{align*}
        Substituting these into the PDE, we find that
        \begin{align*}
            (f(x)g(y))(f'(x)g'(y)) &= (f'(x)g(y))(f(x)g'(y))\\
            f(x)f'(x)g(y)g'(y) &= f(x)f'(x)g(y)g'(y)
        \end{align*}
        Therefore $u(x, y) = f(x)g(y)$ is a valid solution of the PDE, regardless
        of what $f(x)$ or $g(y)$ is.
    \end{solution}
    \section{Problem 5}
    Verify by direct substitution that
    \[
        u_n(x, y) = \sin nx + \sinh ny
    \]
    is a solution of the PDE:
    \[
        u_{xx} + u_{yy} = 0
    \]
    for every $n > 0$.
    \begin{solution}
        We find that
        \begin{align*}
            u_{xx} &= -\sin nx \sinh ny\\
            u_{yy} &= \sin nx \sinh ny\\
        \end{align*}
        and so
        \begin{align*}
            u_{xx} + u_{yy} &= (-\sin nx \sinh ny) + (\sin nx \sinh ny)\\
            &= 0
        \end{align*}
        So, $u_n(x, y) = \sin nx \sinh ny$ is a solution of the PDE $u_{xx} + u_{yy} = 0$
        for every $n > 0$.
    \end{solution}
\end{document}