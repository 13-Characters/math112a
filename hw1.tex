\documentclass{ben}
\title{MATH 112A Homework 1}
\author{Benjamin Tong}
\date{October 10, 2025}

\begin{document}
\maketitle
\section{Monday}
\subsection{Problem 1}
\noindent
Solve the following PDE for $u(x, t)$:
\[
4u_t + 7u_x = 0,\ u(x, 0) = e^x.
\]
Use the method based on directional derivative. Show all the details.
\begin{solution}
    First, we rewrite the PDE as $7u_x + 4u_t = 0$.
    Then, we note that $7u_x + 4u_t = Du \cdot \langle 7, 4 \rangle$. So, the solution to
    $u(x, t)$ must be something of the form
    \[
        u(x, t) = f(4x - 7t)
    \]
    Since our constraint is that $u(x, 0) = e^x$, we find that
    \begin{align*}
        u(x, 0) &= f(4x - 0t)\\
        e^x &= f(4x)\\
        f(4x) &= e^x\\
        f\left( 4 \left( \frac{x}{4} \right) \right) &= e^{\frac{x}{4}}\\
        f(x) &= e^{\frac{x}{4}}
    \end{align*}
    Therefore, $u(x, t) = e^{\frac{(4x - 7t)}{4}}$ is the solution.
\end{solution}
\subsection{Problem 2}
\noindent
Solve the following PDE for $u(x, t)$:
\[
-2u_t + 11u_x = 0, u(x, 0) = \sin 2x
\]
Use the method based on the change of coordinates. Show all the details.
\begin{solution}
    First, we rewrite the PDE as $11u_x - 2u_t = 0$. Then, we define the variables
    $x'$ and $t'$ as
    \begin{align*}
        x' &= -2x + 11t\\
        t' &= 11x + 2t
    \end{align*} 
    Now, we can rewrite both $u_x$ and $u_t$ in terms of $x'$ and $t'$ instead using the chain rule.
    We find that
    \begin{align*} 
        u_x &= \frac{\partial u}{\partial x}\\
        &= \frac{\partial u}{\partial x'} \frac{\partial x'}{\partial x}
        + \frac{\partial u}{\partial t'}\frac{\partial t'}{\partial x}\\
        &= -2 u_{x'} + 11 u_{t'}
    \end{align*}
    and
    \begin{align*} 
        u_t &= \frac{\partial u}{\partial t}\\
        &= \frac{\partial u}{\partial x'} \frac{\partial t'}{\partial t}
        + \frac{\partial u}{\partial t'}\frac{\partial x'}{\partial t}\\
        &= 11 u_{x'} + 2 u_{t'}
    \end{align*}
    Now, we substitute this into the original equation to get
    \begin{align*}
        -2 u_t + 11 u_x &= 0\\
        -2 \left( -2 u_{x'} + 11 u_{t'} \right) + 11 \left( 11 u_{x'} + 2 u_{t'} \right) &= 0\\
        4 u_{x'} - 22 u_{t'} + 121 u_{x'} + 22 u_{t'} &= 0\\
        125 u_{x'} &= 0\\
        u_{x'} &= 0
    \end{align*}
    Integrating this gives us with respect to $x'$ gives us $u = f(t') = f(11x + 2t)$.
    Now, since our constraint is $u(x, 0) = \sin 2x$ this means that
    \begin{align*}
       u(x, 0) &= f(11x + 2(0))\\
       \sin 2x &= f(11x)\\
       f(11x) &= \sin(2x)\\
       f \left( 11 \left( \frac{x}{11} \right) \right) &=
       \sin \left( 2 \left( \frac{x}{11} \right) \right)\\
       f(x) = \sin\left(\frac{2}{11}x\right)
    \end{align*}
    So, $u(x, t) = \sin\left(\frac{2}{11}\left(11x + 2t\right)\right)$ is the solution to the PDE.
\end{solution}
\section{Wednesday}
\subsection{Problem 1}
\noindent
Solve the solution $u(x, t)$ to the equation
\[
u_t + u_x = 1,\ u(x, 0) = 2x^2 - 3x + 5
\]
\begin{solution}
    We define the variables $x'$ and $t'$ as
    \begin{align*}
        x' &= x + t\\
        t' &= -x + t
    \end{align*}
    We find that
    \begin{align*}
        u_t &= \frac{\partial u}{\partial x'} \frac{\partial x'}{\partial t}
        + \frac{\partial u}{\partial t'} \frac{\partial t'}{\partial t}\\
        &= u_{x'} + u_{t'}
    \end{align*}
    and
    \begin{align*}
        u_x &= \frac{\partial u}{\partial x'} \frac{\partial x'}{\partial x}
        + \frac{\partial u}{\partial t'} \frac{\partial t'}{\partial x}\\
        &= u_{x'} - u_{t'}
    \end{align*}
    Integrating this with respect to $x'$ gives us $u = \frac{1}{2}x' + f(t')$
    In terms of $x$ and $t$, $u(x, t) = \frac{1}{2} \left( x + t \right) + f\left( -x + t \right)$.
    Now, since our constraint is $u(x, 0) = 2x^2 - 3x + 5$, it means that
    \begin{align*}
        u(x, 0) &= 2x^2 - 3x + 5\\
        \frac{1}{2}(x + 0) + f(-x + 0) &= 2x^2 - 3x + 5\\
        \frac{1}{2} x + f(-x) &= 2x^2 - 3x + 5\\
        f(-x) &= 2x^2 - \frac{7}{2}x + 5\\
        f(-(-x)) &= 2(-x)^2 - \frac{7}{2}(-x) + 5\\
        f(x) &= 2x^2 + \frac{7}{2}x + 5
    \end{align*}
    Therefore, the solution to the PDE is
    \[
        u(x, t) = \frac{1}{2}(x + t) + \left( 2 (-x + t)^2 + \frac{7}{2} (-x + t) + 5 \right)
    \]
\end{solution}
\subsection{Problem 2}
\noindent
Solve the equation
\[
(1 + t^2)u_t + u_x = 0,\ u(x, 0) = \frac{1}{1 + x^2}
\]
\begin{solution}
    First of all, we find the equation of the characteristic curves. The characteristic curves must
    follow the equation
    \[
        \frac{dt}{dx} = \frac{1 + t^2}{1}
    \]
    This ODE has the solution
    \[
        x = \arctan(t) + C
    \]
    This means that $u(\arctan(t) + C, t)$ will always have the same value regardless of what $t$
    is, and this value is only dependent on $C$. Since $C = x - \arctan(t)$, we say that
    \[u(x, t) = f(x - \arctan(t))\]
    An additional constraint states that $u(x, 0) = \frac{1}{1 + x^2}$, so
    \begin{align*}
        f(x - \arctan(0)) &= \frac{1}{1 + x^2}\\
        f(x) &= \frac{1}{1 + x^2}
    \end{align*}
    Therefore,
    \[
    u(x, t) = \frac{1}{1 + \left(x - \arctan(t)\right)^2}
    \]
    is the solution to the PDE.
\end{solution}
\newpage
\subsection{Problem 3}
\noindent
Solve the PDE $u_x + u_y + u = e^{x + 2y}$ with $u(x, 0) = 0$.
\begin{solution}
    First, we define the variables $x'$ and $y'$ as
    \begin{align*}
        x' &= x + y\\
        y' &= y - x
    \end{align*}
    We find that we can rewrite $u_x$ and $u_y$ in terms of $x'$ and $y'$ like so:
    \begin{align*}
        u_x &= \frac{\partial u}{\partial x'} \frac{\partial x'}{\partial x} + \frac{\partial u}{\partial y'} \frac{\partial y'}{\partial x}\\
        &= 1 u_{x'} - 1 u_{y'}
    \end{align*}
    and
    \begin{align*}
        u_y &= \frac{\partial u}{\partial x'} \frac{\partial x'}{\partial y} + \frac{\partial u}{\partial y'} \frac{\partial y'}{\partial y}\\
        &= 1 u_{x'} + 1 u_{y'}
    \end{align*}
    Now, substituting $u_x$ and $u_y$ with these values we get
    \begin{align*}
        \left( u_{x'} - u_{y'} \right) + \left( u_{x'} + u_{y'} \right) + u &= e^{x + 2y}\\
        2 u_{x'} + u &= e^{x + 2y}
    \end{align*}
    Note that $x + 2y = \frac{3}{2}x' - \frac{1}{2}y'$, so we get the equation
    \begin{align*}
        2 u_{x'} + u &= e^{\left(\frac{3}{2}x' - \frac{1}{2}y'\right)}\\
        &= e^{-\frac{1}{2}y'}e^{\frac{3}{2}x'}
    \end{align*}
    Now, we can solve this like a normal ODE. We find that
    \begin{align*}
        2u_{x'} + u &= e^{-\frac{1}{2}y'}e^{\frac{3}{2}x'}\\
        u_{x'} + \frac{1}{2}u &= \frac{1}{2} e^{-\frac{1}{2}y'}e^{\frac{3}{2}x'}\\
        e^{\frac{1}{2}x'}\left(u_{x'} + \frac{1}{2}u\right)
        &= e^{\frac{1}{2}x'}\left(\frac{1}{2}e^{-\frac{1}{2}y'}e^{\frac{3}{2}x'}\right)\\
        e^{\frac{1}{2}x'} u_{x'} + \frac{1}{2} e^{\frac{1}{2}x'} u
        &= \frac{1}{2}e^{-\frac{1}{2}y'} e^{2x'}\\
        \frac{d}{dx'} \left(e^{\frac{1}{2}x'} u\right)
        &= \frac{1}{2}e^{-\frac{1}{2}y'} e^{2x'}\\
        e^{\frac{1}{2}x'}u &=
        \int \frac{1}{2} e^{-\frac{1}{2}y'}e^{2x'}\ dx\\
        e^{\frac{1}{2}x'}u &=
        \frac{1}{4} e^{\frac{1}{2}y'} e^{2x'} + g(y')\\
        u &= \frac{1}{4} e^{\frac{1}{2}y'}e^{\frac{3}{2}x'} + \frac{g(y')}{e^{\frac{1}{2}x'}}
    \end{align*}
    Substituting in back the values of $x'$ and $y'$ in terms of $x$ and $y$ we get
    \begin{align*}
        u &= \frac{1}{4} e^{\frac{1}{2}y'} e^{\frac{3}{2}x'} + \frac{g(y')}{e^{\frac{1}{2}x'}}\\
        &= \frac{1}{4} e^{\frac{1}{2}(y - x)} e^{\frac{3}{2}(x + y)} + \frac{g(y - x)}{e^{\frac{1}{2}(x + y)}}\\
        &= \frac{1}{4} e^{\frac{1}{2}(y - x) + \frac{3}{2}(x + y)} + \frac{g(y - x)}{e^{\frac{1}{2}(x + y)}}\\
        &= \frac{1}{4} e^{x + 2y} + \frac{g(y - x)}{e^{\frac{1}{2}(x + y)}}
    \end{align*}
    Now, we use account for our constraint $u(x, 0) = 0$. Letting $y = 0$ and solving for $g$, we get
    \begin{align*}
        0 &= \frac{1}{4}e^{x} + \frac{g(-x)}{e^{\frac{1}{2}x}}\\
        -\frac{1}{4}e^{x} &= \frac{g(-x)}{e^{\frac{1}{2}x}}\\
        -\frac{1}{4}e^{\frac{3}{2}x} &= g(-x)\\
        g(-x) &= -\frac{1}{4}e^{\frac{3}{2}x}\\
        g(x) &= -\frac{1}{4}e^{-\frac{3}{2}x}
    \end{align*}
    Now that we have solved for $g$ we find that
    \begin{align*}
        u &= \frac{1}{4}e^{x + 2y} + \frac{g(y - x)}{e^{\frac{1}{2}(x + y)}}\\
        &= \frac{1}{4}e^{x + 2y} + \frac{-\frac{1}{4}e^{-\frac{3}{2}(y - x)}}{e^{\frac{1}{2}(x + y)}}\\
        &= \frac{1}{4}e^{x + 2y} - \frac{1}{4}e^{-\frac{3}{2}(y - x) - \frac{1}{2}(x + y)}\\
        &= \frac{1}{4}e^{x + 2y} - \frac{1}{4}e^{x - 2y}\\
        &= \frac{1}{4}e^x\left( e^{2y} - e^{-2y} \right)\\
        &= \frac{1}{4}e^x \sinh 2y
    \end{align*}
\end{solution}
\section{Friday}
\subsection{Problem 1}
\noindent
For any non-integer real number $\alpha$, derive the binomial formula:
\[
(1 + x)^\alpha = 1 + \alpha x + \frac{\alpha(\alpha - 1)}{2!} x^2
+ \frac{\alpha (\alpha - 1)(\alpha - 2)}{3!}x^3 + \cdots
\]
where the coefficient for $x^n$ is
\[
\frac{\alpha (\alpha - 1) (\alpha - 2) \cdots (\alpha - n + 1)}{n!}
\]
\begin{solution}
    First, we will quickly show that the $n$th derivative of $f(x) = (1 + x)^\alpha$ is equal to
    \[
    \frac{d^n}{dx^n}\left[ (1 + x)^\alpha \right] = \alpha (\alpha - 1) (\alpha - 2) \cdots (\alpha - n + 1) x^{\alpha - n}
    \]
    if $n \geq 1$.\\
    First, let $n = 1$. Then, we have to show that
    $\frac{d}{dx} \left[ (1 + x)^\alpha \right] = \alpha (1 + x)^{\alpha - 1}$.
    If we use the power rule, we can confirm that this equality is true.
    Now, assume that
    \[
    \frac{d^n}{dx^n} \left[ (1 + x)^\alpha \right] = \alpha (\alpha - 1) (\alpha - 2) \cdots (\alpha - n + 1) x^{\alpha - n}
    \]
    is true.
    If we take the derivative on both sides and use the power rule, we find that
    \begin{align*}
        \frac{d}{dx} \left[ \frac{d^n}{dx^n} \left[ (1 + x)^\alpha \right] \right]
        &= \frac{d}{dx} \left[ \alpha (\alpha - 1) (\alpha - 2) \cdots (\alpha - n + 1) x^{\alpha - n} \right]\\
        \frac{d^{n+1}}{dx^{n+1}} \left[ (1 + x)^\alpha \right]
        &= \alpha (\alpha - 1) (\alpha - 2) \cdots (\alpha - n + 1)(\alpha - n)x^{(\alpha - n) - 1}\\
        &= \alpha (\alpha - 1) (\alpha - 2) \cdots (\alpha - n + 1)(\alpha - (n + 1) - 1) x^{\alpha - (n + 1)}\\ 
    \end{align*}
    And so by induction
    $\frac{d^n}{dx^n} \left[ (1 + x)^\alpha \right] = \alpha (\alpha - 1) (\alpha - 2) \cdots (\alpha - n + 1) x^{\alpha - n}$
    is true for all $n \geq 1$.
    
    Now, we can express $f(x) = (1 + x)^\alpha$ as a Taylor series centered at $x = 0$
    \[
    (1 + x)^\alpha = f(0) + \frac{f^{(1)}(0)}{1!} + \frac{f^{(2)}(0)}{2!} + \cdots
    + \frac{f^{(n)}(0)}{n!}
    \]
    Now, substituting $x = 0$ into $f(x)$ we get $f(0) = 1$ and using the derivatives we found
    earlier we find that $f^{(n)} = \alpha (\alpha - 1) \cdots (\alpha - n + 1)$ for all $n \geq 1$.
    Therefore,
    \[
    (1 + x)^\alpha = 1 + \alpha x + \frac{\alpha(\alpha - 1)}{2!} x^2
    + \frac{\alpha (\alpha - 1)(\alpha - 2)}{3!}x^3 + \cdots
    \]
\end{solution}
\subsection{Problem 2}
\noindent
A flexible chain of length $\ell$ is hanging from one end $x = 0$ but oscillates
horizontally. Let the $x$ axis point downward and the $u$ axis point to the right.
Assume that the force of gravity at each point of the chain equals the weight
of the part of the chain below the point and is directed tangentially along the
chain. Assume that the oscillations are small. Find the PDE satisfied by the
chain.
\begin{solution}
    Let $T(x, u)$ be the force of tension tangentially along the string. 
    Due to Newton's second law the following equation must also be met
    \[
    \frac{T(x, u)u_x}{\sqrt{1 + u_x^2}} \Bigg|_{x = x_0}^{x = x_1} = \int_{x_0}^{x_1} \rho u_{tt} dx
    \]
    Since the oscillations are small, and because of the binomial equation we found earlier,
    we can assume $\sqrt{1 + u_x^2} \approx 1$. We can also assume that the horizontal component of
    the tension is negligible compared to the vertical component, meaning that
    \[
        T(x, u) = \rho g (l - x)
    \]
    Where $g$ is the acceleration due to gravity and $\rho$ is the density.
    Now, 
\end{solution}
\end{document}