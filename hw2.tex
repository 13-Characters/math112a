\documentclass{ben}
\title{MATH 112A Homework 2}
\author{Benjamin Tong}
\date{October 17, 2025}

\begin{document}
\maketitle
\section{Monday}
\subsection{Problem 1}
On the sides of a thin rod, heat exchange takes place (obeying Newton's
law of cooling—flux proportional to temperature difference) with a medium of
constant temperature $T_0$. What is the equation satisfied by the temperature $u(x, t)$, neglecting
its variation across the rod?
\begin{solution}
    When we consider the one dimensional case of the heat equation
    $c \rho \frac{\partial x}{\partial t} = \nabla \cdot (\kappa \Delta u)$, the equation becomes
    \[
    c \rho u_t = \kappa u_{xx}
    \]
\end{solution}
\subsection{Problem 2}
Consider heat flow in a long circular cylinder where the temperature depends only on $t$
and on the distance $r$ to the axis of the cylinder. Here $r = \sqrt{x^2 + y^2}$ is the cylindrical
coordinate. From the three-dimensional heat equation derive the equation
\[
    u_t = \kappa \left( u_{rr} + \frac{u_r}{r} \right)
\]
\begin{solution}
    We know that the three-dimensional heat equation is the equation
    \[
        u_t = k(u_{xx} + u_{yy} + u_{zz})
    \]
    Converting this into cylindrical coordinates, we get
    \[
        u_t =
        \frac{1}{r} \frac{\partial }{\partial r} \left( r \frac{\partial u}{\partial r} \right)
        + \frac{1}{r^2} \frac{\partial^2 u}{\partial \theta^2} + \frac{\partial^2 u}{\partial z^2}
    \]
    We know that since $u$ only depends on $r$ and $t$,
    $\frac{\partial^2 u}{\partial \theta}$ and $\frac{\partial^2 u}{\partial z^2}$
    are both equal to $0$. So, this equation is simplified to
    \[
        u_t = k \left[ \frac{1}{r} \frac{\partial }{\partial r} \left( r \frac{\partial u}{\partial r} \right) \right]
    \]
    Simplifying further this we get
    \begin{align*}
        u_t &= k \left[ \frac{1}{r} \frac{\partial }{\partial r} \left( r \frac{\partial u}{\partial r} \right) \right]\\
        &= \frac{k}{r} \left( \frac{\partial u}{\partial r} + r \frac{\partial^2 u}{\partial r^2} \right)\\ 
        &= k \left( \frac{1}{r} \frac{\partial u}{\partial r} + \frac{\partial^2 u}{\partial r^2} \right)\\
        &= k \left( u_{rr} + \frac{1}{r} u_{r} \right)
    \end{align*}
\end{solution}
\subsection{Problem 3}
If $\vec{f} (\vec{x})$ is continuous and $|\vec{f} (\vec{x})| \leq \frac{1}{|\vec{x}|^3 + 1}$ for
all $\vec{x}$, show that
\[
    \iiint_{\R^3} \operatorname{div}\left(\vec{f} (\vec{x}) \right) d \vec{x} = 0
\]
\section{Wednesday}
\subsection{Problem 1}
By trial and error, find a solution of the diffusion equation $u_t = u_{xx}$ with the initial
conditional $u(x, 0) = x^2$.
\subsection{Problem 2}
A homogeneous body occupying the solid region $D$ is completely insulated.
Its initial temperature is $f(\vec{X})$. Find the steady-state temperature that it reaches after
a long time. (Hint: No heat is gained or lost.)
\begin{solution}
    The steady state temperature would be
    \[
    \frac{\iiint_{D} f(\vec{X}) dV}{\iiiint_D dV} 
    \]
    since we expect that after an infinite amount of time that the temperature would average out
\end{solution}
\subsection{Problem 3}
A rod occupying the interval $0 \leq x \leq \ell$ is subject to the heat source $f(x) = 0$ for
$0 < x < \frac{\ell}{2}$, and $f(x) = H$ for $\frac{\ell}{2} < x < \ell$ where $H > 0$. The rod has
physical constants $c = \rho = \kappa = 1$, and its ends are kept at zero temperature.
Find the steady-state temperature $T(x)$ of the rod.
Here $T(x)$ is a continuously differentiable function.
\section{Friday}
\subsection{Problem 1}
Solve the boundary problem $u'' = 0$ for $0 < x < 1$ with $u'(0) + ku(0) = 0$ and
$u'(1) \pm ku(1) = 0$. Do the $+$ and $-$ cases separately. What is special about the case $k = 2$?
\subsection{Problem 2}
Consider the Neumann problem
\begin{align*}
    \Delta u = f(x, y, z),&\qquad \text{in $D$.}\\
    \frac{\partial u}{\partial n} = 0,&\qquad \text{on $\partial D$.}
\end{align*}
\begin{enumerate}[label=(\arabic*)]
    \item What can we surely add to any solution to get another solution?
    So we don't have uniqueness.
    \item Use the divergence theorem and the PDE to show that
    \[
        \iiint_D f(x, y, z)\ dx\ dy\ dz = 0
    \]
    \item Can you give a physical interpretation part (1) and/or (2)
    for either heat flow or diffusion?
\end{enumerate}
\end{document}