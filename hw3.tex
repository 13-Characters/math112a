\documentclass{ben}
\title{MATH 112A Homework 3}
\author{Benjamin Tong}
\date{October 24, 2025}

\begin{document}
\maketitle
\section{Monday}
\subsection{Problem 1}
What is the type of each of the following equations?
\begin{enumerate}
    \item $u_{xx} - u_{xy} + 2 u_y + u_{yy} - 3 u_{yx} + 4u = 0$
    \item $9u_{xx} + 6u_{xy} + u_{yy} + u_{x}$
\end{enumerate}
\begin{multipart}
    \item This equation simplifies to $u_{xx} - 4u_{xy} + u_{yy} + 2 u_y + 4u = 0$.
    Since $(-2)^2 < (1)(1)$, this equation is elliptic.
    \item Since $3^2 = (9)(1)$, this equation is parabolic.
\end{multipart}
\subsection{Problem 2}
Find the regions in the $xy$ plane where the equation
\[
  (1 + x) u_{xx} + 2xy u_{xy} - y^{2} u_{yy} = 0
\]
is elliptic, hyperbolic or parabolic. Sketch them.
\begin{solution}
  This equation is parabolic only on the line $y = 0$. Everywhere else, the
  equation is hyperbolic.
  \begin{center}
    \includegraphics[width=5.03in]{hw3graph.png}
  \end{center}
\end{solution}
\newpage
\subsection{Problem 3}
What is the type of the equation
\[
  u_{xx} - 4 u_{xy} + 4 u_{yy} = 0
\]
Show by direct substitution that $u(x, y) = f(y + 2x) + xg(y + 2x)$ is a solution for arbitrary
functions $f$ and $g$.
\begin{solution}
  This equation is parabolic, as $(-2)^2 = (1)(4)$. Now, we will find the derivatives
  $u_{xx}$, $u_{xy}$, and $u_{yy}$ for the equation $u(x, y) = f(y + 2x) + xg(y + 2x)$.\\
  We find that
  \begin{align*}
    u_{x} &= 2 f'(y + 2x) + \left( g(y + 2x) + 2x g'(y + 2x) \right)\\
    u_{y} &= f'(y + 2x) + xg'(y + 2x)
  \end{align*}
  Differentiating this again we get
  \begin{align*}
    u_{xx} &= 4 f''(y + 2x) + 4 g'(y + 2x) + 4x g''(y + 2x)\\
    u_{xy} &= 2 f''(y + 2x) + g'(y + 2x) + 2x g''(y + 2x)\\
    u_{yy} &= f''(y + 2x) + x g''(y + 2x)
  \end{align*}
  Substituting in these equations we get that
  \begin{align*}
    u_{xx} - 4u_{xy} + 4u_{yy} &= \left( 4 f''(y + 2x) + 4 g'(y + 2x)
                               + 4x g''(y + 2x) \right)\\
                               & - 4 \left( 2 f''(y + 2x) + g'(y + 2x)
                               + 2x g''(y + 2x) \right)\\
                               & + 4 \left( f''(y + 2x) + x g''(y + 2x) \right)\\
                               &= \left( 4 f''(y + 2x) + 4 g'(y + 2x)
                               + 4x g''(y + 2x) \right)\\
                               & + \left( -8 f''(y + 2x) - 4 g'(y + 2x)
                               - 8x g''(y + 2x) \right)\\
                               & + \left( 4 f''(y + 2x) + 4x g''(y + 2x) \right)\\
                               &= (4 + (-8) + 4) f''(y + 2x) + (4 + (-4)) g'(y + 2x)
                               + (4 + (-8) + 4) x g''(y + 2x)\\
                               &= 0 f''(y + 2x) + 0 g'(y + 2x) + 0 x g''(y + 2x)\\
                               &= 0 + 0 + 0\\
                               &= 0
  \end{align*}
  
\end{solution}
\section{Wednesday and Friday}
\subsection{Problem 1}
Solve
\[
  u_{tt} = c^2 u_{xx},\ u(x, 0) = e^{x},\ u_t(x, 0) = \sin x
\]
\begin{solution}
    To solve this, we use D' Alembert's formula, which is
    \[
        u(x, t) = \frac{1}{2} \left(\phi(x + ct) + \phi(x - ct)\right) + \frac{1}{2c} \int_{x - ct}^{x + ct} \psi(s)\ ds
    \]
    where $\phi(x) = u(x, 0) = e^x$ and $\psi(x) = u_t(x, 0) = \sin x$
    Substituting these equations we get
    \begin{align*}
        u(x, t) &= \frac{1}{2} \left( e^{x + ct} - e^{x - ct} \right) + \frac{1}{2c} \int_{x - ct}^{x + ct} \sin(s) ds\\
        &= \frac{1}{2} \left( e^{x + ct} - e^{x - ct} \right) + \frac{1}{2c} \left( -\cos(x + ct) - (- \cos(x + ct)) \right)\\
        &= \frac{1}{2} \left( e^{x + ct} - e^{x - ct} \right) + \frac{1}{2c} \left( \cos(x - ct) - \cos(x + ct) \right)
    \end{align*}
\end{solution}
\subsection{Problem 2}
Solve
\[
  u_{tt} = c^2 u_{xx},\ u(x, 0) = \ln(1 + x^2),\ u_t(x, 0) = 4 + x
\]
\begin{solution}
    To solve this, we use D' Alembert's formula, which is
    \[
        u(x, t) = \frac{1}{2}(\phi(x + ct) + \phi(x - ct)) + \frac{1}{2c} \int_{x - ct}^{x + ct} \psi(s) ds
    \]
    where $\phi(x) = u(x, 0) = \ln(1 + x^2)$ and $\psi(x) = u_t(x, 0) = 4 + x$. Substituting
    these equations we get
    \begin{align*}
        u(x, t) &= \frac{1}{2}\left( \ln(1 + (x + ct)^2) + \ln(1 + (x - ct)^2) \right) + \frac{1}{2c} \int_{x - ct}^{x + ct} 4 + s\ ds\\
        &= \frac{1}{2} \left( \ln \left((1 + (x + ct)^2)(1 + (x - ct)^2)\right)\right) + \frac{1}{2c} \left( (4 (x + ct) + \frac{1}{2}(x + ct)^2) - (4 (x - ct) + \frac{1}{2}(x - ct)^2)\right)\\
        &= \frac{1}{2} \left( \ln \left(1 + (x - ct)^2 + (x + ct)^2 + (x - ct)^2(x + ct)^2 \right) \right) + \frac{1}{2c} \left( 8ct + xct \right)
    \end{align*}
\end{solution}
\subsection{Problem 3}
(The hammer blow) Consider the 1D wave equation:
\[
  \begin{cases}
    u_{tt} = c^2 u_{xx}, & -\infty < x < \infty,\ t > 0;\\
    u(x, 0) = \phi(x), u_{t}(x, 0) = \psi (x), & -\infty < x < \infty.
  \end{cases}
\]
Let $\phi(x) = 0$ and $\psi(x) = 1$ for $|x| < a$ and $\psi(x) = 0$ for $|x| \geq a$. Sketch
the string profile ($u$ versus $x$) at each of the successive instants $t = \frac{a}{2c}$,
$\frac{a}{c}$, $\frac{3a}{2c}$, $\frac{2a}{c}$, $\frac{5a}{c}$ 

\begin{solution}
    First, we use D'Alembert's formula to solve the equation. The formula states that
    \[
    u(x, t) = \frac{1}{2} \left(\phi(x + ct) - \phi(x - ct) \right)
    + \frac{1}{2c} \int_{x - ct}^{x + ct} \psi(s)\ ds
    \]
    We know by definition that $\phi(x) = 0$, so,
    \begin{align*}
        u(x, t) &= \frac{1}{2c} \int_{x - ct}^{x + ct} \psi(s) ds\\
        &= \frac{1}{2c} \operatorname{len} \{(x - ct, x + ct) \cap (-a, a)\}
    \end{align*}
    For the $t = \frac{a}{2c}$ case, we find that the maximum value of $u$ must be equal to $a$,
    and this happens when $-\frac{a}{2} < x < \frac{a}{2}$. We also find that $u$ is not equal
    to zero on the interval $-\frac{3a}{2} < x < \frac{3a}{2}$\\
    For the other cases, since the interval $(-a, a)$ is smaller than the interval
    $(x - ct, x + ct)$, the maximum value of $u$ must be $2a$ since that is the length of $(-a, a)$,
    and this happens when $-a + ct < x < a - ct$. We also find that $u$ is not equal to
    zero on the interval $-a - ct < x < a + ct$.

    With this information, we find that the string profiles are:
    \newpage
    \begin{center}
        \includegraphics[height=7in]{hw3graph2.png}
    \end{center}
\end{solution}
\newpage
\subsection{Problem 4}
In P3, find the greatest displacement, $\max_{x} u(x, t)$, as a function of $t$.
\begin{solution}
  When $t$ is fixed, the maximum value of $u(x, t)$ is the length of the intersection of the
  open sets $(x - ct, x + ct)$ and $(-a, a)$. This happens when one of the sets is contained inside
  the other, and when this happens the length of the intersection is just the length of the set that
  is smaller. So,
  \[\max_x u(x, t) = \min(2a, 2ct)\]
\end{solution}
\subsection{Problem 5}
Consider the 1D wave equation:
\[
  \begin{cases}
    u_{tt} = c^2 u_{xx}, & -\infty < x < \infty,\ t > 0;\\
    u(x, 0) = \phi(x), u_{t}(x, 0) = \psi(x), & -\infty < x < \infty.
  \end{cases}
\]
If both $\phi$ and $\psi$ are odd functions of $x$, show that the solution $u(x, t)$ of the wave
equation is also odd in $x$ for all $t$.
\begin{solution}
  First, we find the solution is equal to D' Alembert's equation, which is
  \[
    u(x, t) = \frac{1}{2} \left( \phi(x + ct) + \phi(x - ct) \right)
    + \frac{1}{2c} \int_{x - ct}^{x + ct} \psi(s)\ ds
  \]
  Now, we want to show that $u$ is odd in $x$ for all $t$, that is, $u(-x, t) = u(x, t)$
  for all $t$. We are given only the information that $\phi(-x) = -\phi(x)$ and
  $\psi(-x) = -\psi(x)$. We find that
  \begin{align*}
    u(-x, t) &= \frac{1}{2} \left( \phi((-x) + ct) + \phi((-x) - ct) \right)
    + \frac{1}{2c} \int_{(-x) - ct}^{(-x) + ct} \psi(s)\ ds\\
    &= \frac{1}{2} \left( -(\phi(x - ct)) - (\phi(x + ct)) \right)
    + \frac{1}{2c} - \int_{x + ct}^{x - ct} \psi(s)\ ds\\
    &= -\frac{1}{2} \left( \phi(x - ct) + \phi(x + ct) \right) - \frac{1}{2c} \int_{x - ct}^{x + ct} \psi(s)\ ds\\
    &= -\frac{1}{2} \left( \phi(x - ct) + \phi(x + ct) \right) - \frac{1}{2c} \int_{x - ct}^{x + ct} \psi(s)\ ds\\
    &\neq -u(x, t)
  \end{align*}
  So, $\psi$ needs to be even in order for $u(x, t)$ to be true
\end{solution}
\end{document}
