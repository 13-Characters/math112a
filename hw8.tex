\documentclass{ben}
\DeclareMathOperator{\erf}{erf}

\title{MATH 112A Homework 8}
\date{November 28, 2025}
\author{Benjamin Tong}
\begin{document}
  \maketitle
  \section{Problem 1}
  Solve the diffusion equation:
  \[
    \begin{cases}
      u_{t} = ku_{xx}, & x > 0, t > 0\\
      u(x, 0) = \sinh x, & x > 0\\
      u(0, t) = 0, & t \ge 0.
    \end{cases}
  \]
  \begin{solution}
    First, note that $\sinh x$ is odd. This means that the function is equal to its
    own odd extension, and so we are free to use the whole line formula:
    \[
      u(x, t) = \frac{1}{\sqrt{4\pi kt}} \int_{-\infty}^{\infty}
      \left(e^{\frac{-(x - y)^2}{4kt}}\right) \sinh y \ dy
    \]
    Now, we note that $\sinh y = \frac{e^y - e^{-y}}{2}$ and we get that
    \begin{align*}
      u(x, t) &= \frac{1}{\sqrt{4\pi kt}} \int_{-\infty}^{\infty}
      \left(e^{\frac{-(x - y)^2}{4kt}}\right) \sinh y \ dy\\
      &= \frac{1}{\sqrt{4\pi kt}} \int_{-\infty}^{\infty}
      \left(e^{\frac{-(x - y)^2}{4kt}}\right) \left(\frac{e^y}{2} - \frac{e^{-y}}{2}\right)\ dy\\
      &= \frac{1}{2\sqrt{4\pi kt}} \int_{-\infty}^{\infty} e^{\frac{-(x - y)^2}{4kt} + y}
      - \frac{1}{2\sqrt{4\pi kt}} \int_{-\infty}^{\infty} e^{\frac{-(x - y)^2}{4kt} - y}
    \end{align*}
    Now, note that the exponents on both integrals are of the form
    \[
      E = \frac{-(x - y)^2}{4kt} \pm y
    \]
    Rewriting this exponent, we get
    \begin{align*}
      E &= \frac{-(x - y)^2}{4kt} \pm y\\
        &= \frac{-x^2 + (2x \pm 4kt)y - y^2}{4kt}\\
        &= \frac{-x^2 + (x^2 \pm 4ktx + 4 k^2 t^2) - (y - x \mp 2kt)^2}{4kt}\\
        &= (kt \pm x) - \frac{(y - x \mp 2kt)^2}{4kt}
    \end{align*}
    and thus we can rewrite the integral like this:
    \[
    u(x, t) = \frac{e^{kt + x}}{2 \sqrt{4 \pi kt}} \int_{-\infty}^{\infty}
    e^{\frac{-(y - x - 2kt)^2}{4kt}}\ dy
    - \frac{e^{kt - x}}{2 \sqrt{4 \pi kt}}
    e^{\frac{-(y - x + 2kt)^2}{4kt}}\ dy
    \]
    Now, let $\alpha = \frac{y - x - 2kt}{\sqrt{4kt}}$ and $\beta = \frac{y - x + 2kt}{\sqrt{4kt}}$
    and substitute these into the integral. Doing this, we get
    \begin{align*}
      u(x, t) &= \frac{e^{kt + x}}{2 \sqrt{4 \pi kt}} \int_{-\infty}^{\infty}
      e^{-\alpha^2}\ (\sqrt{4kt} d \alpha
      - \frac{e^{kt - x}}{2 \sqrt{4 \pi kt}} \int_{-\infty}^{\infty}
      e^{-\beta^2}\ (\sqrt{4kt} d \beta)\\
      &= \frac{e^{kt + x}}{2\sqrt{\pi}} \int_{-\infty}^{\infty} e^{-\alpha^2}\ d\alpha
      - \frac{e^{kt - x}}{2\sqrt{\pi}} \int_{-\infty}^{\infty} e^{-\beta^2}\ d\beta\\
      &= \frac{e^{kt + x}}{2\sqrt{\pi}} (\sqrt{\pi}) - \frac{e^{kt - x}}{2\sqrt{\pi}} (\sqrt{\pi})\\
      &= \frac{e^{kt + x}}{2} - \frac{e^{kt - x}}{2}\\
      &= e^{kt} \left(\frac{e^x}{2} - \frac{e^{-x}}{2}\right)\\
      &= e^{kt} \sinh(x)
    \end{align*}
    Now, if you don't believe that we got the correct answer by using the whole line formula, we
    can check that this is the solution by computing
    $u_{xx}$ and $u_t$. We find that
    \begin{align*}
      u &= \sinh(x) \cdot e^{kt}\\
      u_{x} &= \cosh(x) \cdot e^{kt}\\
      u_{xx} &= \sinh(x) \cdot e^{kt}\\
      u_t &= k \cdot \sinh(x) \cdot e^{kt}
    \end{align*}
    This satisfies $u_{xx} = ku_t$. Next, we check if the initial condition is satisfied.
    Since $e^{kt} = 1$ if $t = 0$ regardless of what $k$ is, this means that $u(x, 0) = \sinh(x)$ is
    satisfied. Finally, since $\sinh(0) = 0$, $e^{kt}\sinh(0) = 0$ for all $k, t$. Therefore,
    the boundary condition is satisfied and this is a valid solution.
 \end{solution}
  \section{Problem 2}
  Solve the diffusion equation:
  \[
    \begin{cases}
      u_{t} = k u_{xx}, & x > 0, t > 0\\
      u(x, 0) = x, & x > 0\\
      u_{x}(0, t) = 0, & t \ge 0.
    \end{cases}
  \]
  \begin{solution}
  First, we use the equation
  \[
  u(x, t) = \frac{1}{\sqrt{4 \pi kt}} \int_{0}^{\infty}
  \left(e^{\frac{-(x - y)^2}{4kt}} + e^{\frac{-(x + y)^2}{4kt}}\right)y \ dy
  \]
  then we split the integral into two parts:
  \[
  u(x, t) = \frac{1}{\sqrt{4 \pi kt}} \left( \int_{0}^{\infty} y e^{\frac{-(x - y)^2}{4kt}}\ dy
  + \int_{0}^{\infty} y e^{\frac{-(x + y)^2}{4kt}}\ dy \right)
  \]
  We now need to solve the integrals $\int_{0}^{\infty} y e^{-\frac{(x - y)^2}{4kt}}\ dy$ and
  $\int_{0}^{\infty} y e^{-\frac{(x + y)^2}{4kt}}\ dy$
  First, we solve the integral $\int_{0}^{\infty} y e^{\frac{-(x - y)^2}{4kt}}\ dy$
  Before we integrate by substitution, we substitute
  $y = \frac{(y - x)}{\sqrt{4kt}} \cdot \sqrt{4kt} + x$., and the exponent is
  $\frac{-(y - x)^2}{4kt}$. Now the integral is
  \[
  \int_{0}^{\infty} \left( \frac{(y - x)}{\sqrt{4kt}} \cdot \sqrt{4kt} + x \right)
  e^{\frac{-(y - x)^2}{4kt}}\ dy
  \]
  Now we make the substitution $\alpha = \frac{(y - x)}{\sqrt{4kt}}$ and we get
  \begin{align*}
    &= \int_{\frac{-x}{\sqrt{4kt}}}^{\infty} \left(\sqrt{4kt} \alpha + x\right) e^{-\alpha^2}\ (\sqrt{4kt} d\alpha)\\
    &= \int_{\frac{-x}{\sqrt{4kt}}}^{\infty} \left(4kt\alpha + \sqrt{4kt}x\right) e^{-\alpha^2}\ d\alpha\\
    &= 4kt \int_{\frac{-x}{\sqrt{4kt}}}^{\infty} \alpha e^{-\alpha^2}\ 
    + \sqrt{4kt} x \int_{\frac{-x}{\sqrt{4kt}}}^{\infty} e^{-\alpha^2}\ d\alpha\\
    &= 4kt \Bigg(-\frac{1}{2}e^{-\alpha^2}\Bigg|_{\frac{-x}{\sqrt{4kt}}}^{\infty} \Bigg)
    + x \sqrt{4kt\pi} \left(\frac{1}{2} + \frac{1}{2} \erf\left(\frac{x}{\sqrt{4kt}}\right)\right)\\
    &= 2kt\left(e^{\frac{-x^2}{4kt}}\right) + x\sqrt{kt\pi} \left(1 + \erf \left(\frac{x}{\sqrt{4kt}}\right)\right)
  \end{align*}
  Now, we find the integral $\int_{0}^{\infty} y e^{-\frac{(x + y)^2}{4kt}}$. Like last time, we
  make the substitution $y = \frac{y + x}{\sqrt{4kt}} \cdot \sqrt{4kt} - x$. We get
  \[
    \int_{0}^{\infty} \left(\frac{y + x}{\sqrt{4kt}} \cdot \sqrt{4kt} - x\right)
    e^{\frac{-(y + x)^2}{4kt}}\ dy
  \]
  Now, let $\beta = \frac{y + x}{\sqrt{4kt}}$ and we get
  \begin{align*}
    &= \int_{\frac{x}{\sqrt{4kt}}}^{\infty}
    \left(\sqrt{4kt} \beta - x\right)e^{-\beta^2}\ (\sqrt{4kt} d\beta)\\
    &= \int_{\frac{x}{\sqrt{4kt}}}^{\infty}
    \left(4kt\beta-\sqrt{4kt}x\right)e^{-\beta^2}\ d\beta\\
    &= 4kt \int_{\frac{x}{\sqrt{4kt}}}^{\infty} \beta e^{-\beta^2}
    - \sqrt{4kt}x \int_{\frac{x}{\sqrt{4kt}}}^{\infty} e^{-\beta^2}\\
    &= 4kt \left(-\frac{1}{2}e^{-\beta^2} \Bigg|_{\frac{x}{\sqrt{4kt}}}^{\infty}\right)
    -\sqrt{4kt}x \left(\frac{1}{2}-\frac{1}{2}\erf \left(\frac{x}{\sqrt{4kt}}\right)\right)\\
    &= 2kt \left(e^{\frac{-x^2}{4kt}}\right)
    + x\sqrt{kt \pi}\left(\erf\left(\frac{x}{4kt}\right) - 1\right)
  \end{align*}
  When we substitute these values into the equation for $u(x, t)$ we get
  \begin{align*}
    u(x, t) &= \frac{1}{\sqrt{4\pi kt}} \left(
      \left(
        2kt \left(e^{-\frac{-x^2}{4kt}}\right)
        + x \sqrt{kt\pi} \left(1 + \erf \left(\frac{x}{4kt}\right)\right)
      \right)
      + \left(
        2kt \left(e^{-\frac{x^2}{4kt}}\right)
        + x \sqrt{kt\pi} \left(\erf \left(\frac{x}{4kt} - 1\right)\right)
      \right)
    \right)\\
    &= \frac{1}{\sqrt{4\pi kt}} \left(
      4kt \left(e^{\frac{-x^2}{4kt}}\right)
      + x \sqrt{kt\pi} \left(
        \left(1 + \erf\left(\frac{x}{4kt}\right)\right)
        + \left(\erf\left(\frac{x}{4kt}\right) - 1\right)
      \right)
    \right)\\
    &= \frac{1}{\sqrt{4\pi kt}} \left(
      4kt \left(e^{\frac{-x^2}{4kt}}\right)
      + 2x \sqrt{kt \pi} \erf \left(\frac{x}{4kt}\right)
    \right)\\
    &= \frac{\sqrt{4kt}}{\sqrt{\pi}} e^{\frac{-x^2}{4kt}} + x \erf \left(\frac{x}{4kt}\right)
  \end{align*}
  \end{solution}
  \newpage
  \section{Problem 3}
  Consider the following problem with a Robin boundary condition:
  \[
    \begin{cases}
      u_{t} = ku_{xx}, & 0 < x < \infty, t > 0\\
      u(x, 0) = x, & 0 < x < \infty\\
      u_{x}(0, t) - 2u(0, t) = 0, & t \ge 0.
    \end{cases}
  \]
  The purpose of this exercise is to verify the solution formula for this equation. Let $f(x) = x$
  for $x > 0$, let $f(x) = x + 1 - e^{2x}$ for $x < 0$, and let
  \[
    v(x, t) = \frac{1}{\sqrt{4 \pi kt}} \int_{-\infty}^{\infty} e^{-\frac{(x - y)^{2}}{4kt}}
    f(y)\ dy
  \]
  \begin{enumerate}
    \item What PDE and initial condition does $v(x, t)$ satisfy for $x \in (-\infty, \infty)$?
    \item Let $w = v_{x} - 2v$. What PDE and initial condition does $w$ satisfy for
    $-\infty < x < \infty$?
    \item Show that $f'(x) - 2f(x)$ is an odd function (for $x \neq 0$).
    \item Show that $w$ is an odd function of $x$.
    \item Assume uniqueness, deduce that $v(x, t)$ solves the equation for $x > 0$.
  \end{enumerate}
  \begin{solution}
    \begin{enumerate}
      \item Because the formula for $v(x, t)$ obeys the formula for the heat equation 
      on the whole line, this means that $v$ is the solution to the following
      initial value problem:
      \[
        \begin{cases}
          v_{t} = kv_{xx},& -\infty < x < \infty\\
          v(x, 0) = f(x)
        \end{cases}
      \]
      \item $w$ satisfies the heat equation since $v$ and $v_x$ are solutions and $v_x - 2_v$
      is a linear combination of the solutions $v$ and $v_x$. We can also show explicitly
      that $w_t = w_xx$ by showing that
      \begin{align*}
        w_t &= (v_{x} - 2v)_{t}\\
            &= v_{xt} - 2v_{t}\\
            &= v_{tx} - 2v_{t}\\
            &= (v_{t})_{x} - 2(v_{t})\\
            &= (v_{xx})_{x} - 2(v_{xx})\\
            &= w_{xx}
      \end{align*}
      \item First of all, we calculate the value of $f'(x) - 2f(x)$. We find that if
      $x > 0$ then
      \[
        f'(x) - 2f(x) = 1 - 2x
      \]
      and if $x < 0$ then
      \begin{align*}
        f'(x) - 2 f(x) &= (1 - 2e^{2x}) - 2(x + 1 - e^{2x})\\
                       &= 1 - 2e^{2x} - 2x - 2 + 2e^{2x}\\
                       &= -1 - 2x
      \end{align*}
      without loss of generality, assume that $x > 0$. Then,
      \begin{align*}
        -f(-x) &= -1 -2(-x)\\
               &= -1 + 2x\\
               &= -(1 - 2x)\\
               &= -f(x)
      \end{align*}
      and so $f'(x) - 2f(x)$ is odd.
      \item Since $w(x, 0) = f'(x) - 2f(x)$, and $f'(x) - 2f(x)$ is odd, then $w$ must be odd
      with respect to $x$.
      \item Since $w$ is odd, we find that $w$ also obeys the following initial value problem
      \[
        \begin{cases}
          w_{t} = kw_{xx} & 0 < x < \infty, t > 0\\
          w(x, 0) = f'(x) - 2 f(x) & 0 < x < \infty, t > 0\\
          w(0, t) = 0 & 0 < \infty, t > 0
        \end{cases}
      \]
      Since $w(0, t) = 0$ is true, this also means that $v_{x}(0, t) - 2v(0, t) = 0$.
      Since $v(x, 0) = x$ for $0 < x < \infty$ and $v$ satisfies the heat equation, $v$ satisfies
      the inital value problem with a
      Robin boundary condition shown at the beginning of the problem.
    \end{enumerate}
  \end{solution}
  \section{Problem 4}
  Solve
  \[
    \begin{cases}
      u_{tt} = 4u_{xx}, & 0 < x < \infty,\\
      u(x, 0) = 1, u_{t}(x, 0) = 0, x > 0,\\
      u(0, t) = 0, t \ge 0.
    \end{cases}
  \]
  Using the reflection method. This solution has a singularity; find its location.
  \begin{solution}
    In order to solve this equation, we will use the formula for the half-line.
    If $x > 2t$, then we use the formula
    \[
      u = \frac{1}{2} \left[ \phi(x + ct) + \phi(x - ct) \right]
      + \frac{1}{2c} \int_{x - ct}^{x + ct} \psi(s)\ ds
    \]
    with $\phi(x) = u(x, 0)$, $\psi(x) = u_{t}(x, 0)$, and $c = 2$.
    We find that the solution is equal to
    \begin{align*}
      u &= \frac{1}{2} \left[ 1 + 1 \right] + \frac{1}{2(2)} \int_{x - 2t}^{x + 2t} 0\ ds\\
        &= \frac{1}{2} (2) + 0\\
        &= 1
    \end{align*}
    If $x < 2t$, then we use the formula
    \[
      u = \frac{1}{2} \left[ \phi(ct + x) - \phi(ct - x) \right]
      + \frac{1}{2c} \int_{ct - x}^{ct + x} \phi(s)\ ds
    \]
    We find that the solution is equal to
    \begin{align*}
      u &= \frac{1}{2} \left[ 1 - 1 \right] + \frac{1}{2(2)} \int_{2t - x}^{2t + x} 0\ ds\\
        &= \frac{1}{2} (0) + 0\\
        &= 0
    \end{align*}
    So, the solution $u(x, t)$ is equal to
    \[
      u(x, t) =
      \begin{cases}
        1 & \text{if } x > 2t\\
        0 & \text{if } x < 2t
      \end{cases}
    \]
    With this formula it is clear that there is a singularity at $x = 2t$.
  \end{solution}
  \section{Problem 5}
  Consider the wave equation
  \[
    \begin{cases}
      u_{tt} = u_{xx}, & 0 < x < 1, t > 0.\\
      u(x, 0) = x^{2}(1 - x), u_{t}(x, 0) = (1 - x)^{2}, & 0 < x < 1.\\
      u(0, t) = u(1, t) = 0, & t \ge 0.
    \end{cases}
  \]
  Find $u(\frac{2}{3}, 2)$ and $u(\frac{1}{4}, \frac{7}{2})$.
  \begin{solution}
    Let $\phi_{\text{odd}}$ and $\psi_{\text{odd}}$ be the odd extensions of $x^{2}(1 - x)$
    and $(1 - x)^2$, respectively, with a period of 1
    We find that
    \begin{align*}
      u \left( \frac{2}{3}, 2 \right)
      &= \frac{1}{2} \left[ \phi_{\text{odd}}\left( \frac{2}{3} + 2 \right)
      + \phi_{\text{odd}}\left( \frac{2}{3} - 2 \right)\right]
      + \int_{\frac{2}{3} - 2}^{\frac{2}{3} + 2} \psi_{\text{odd}}(s)\ ds\\
      &= \frac{1}{2} \left[ \phi_{\text{odd}}\left( \frac{8}{3} \right)
      + \phi_{\text{odd}}\left( -\frac{4}{3} \right)\right]
      + \int_{-\frac{4}{3}}^{\frac{8}{3}} \psi_{\text{odd}}(s)\ ds\\
      &= \frac{1}{2} \left[ -\phi_{\text{odd}}\left( \frac{4}{3} \right)
      - \phi_{\text{odd}}\left( -\frac{2}{3} \right)\right]
      + \int_{-\frac{4}{3}}^{\frac{8}{3}} \psi_{\text{odd}}(s)\ ds\\
      &= \frac{1}{2} \left[ \phi_{\text{odd}}\left( \frac{2}{3} \right)
      + \phi_{\text{odd}}\left( \frac{2}{3} \right)\right]
      + \int_{-\frac{4}{3}}^{\frac{8}{3}} \psi_{\text{odd}}(s)\ ds\\
      &= \frac{1}{2} \left[ \phi_{\text{odd}}\left( \frac{2}{3} \right)
      + \phi_{\text{odd}}\left( \frac{2}{3} \right)\right]\\
      &+ \left(
        \int_{-\frac{4}{3}}^{-1} \psi_{\text{odd}} (s)\ ds
        + \int_{-1}^{0} \psi_{\text{odd}} (s)\ ds
        + \int_{0}^{1} \psi_{\text{odd}} (s)\ ds
        + \int_{1}^{2} \psi_{\text{odd}} (s)\ ds
        + \int_{2}^{\frac{8}{3}} \psi_{\text{odd}} (s)\ ds
      \right)\\
      &= \frac{1}{2} \left[ \phi_{\text{odd}}\left( \frac{2}{3} \right)
      + \phi_{\text{odd}}\left( \frac{2}{3} \right)\right]\\
      &+ \left(
        \int_{\frac{2}{3}}^{1} \psi (s)\ ds
        + \int_{0}^{1} -\psi (s)\ ds
        + \int_{0}^{1} \psi (s)\ ds
        + \int_{0}^{1} -\psi (s)\ ds
        + \int_{0}^{\frac{1}{3}} \psi (s)\ ds
      \right)\\
      &= \frac{1}{2} \left[ \phi_{\text{odd}}\left( \frac{2}{3} \right)
      + \phi_{\text{odd}}\left( \frac{2}{3} \right)\right] + \left(
        \int_{0}^{\frac{2}{3}} \psi (s)\ ds + \int_{\frac{2}{3}}^{1} \psi (s)\ ds
        + \int_{0}^{1} -\psi (s)\ ds \right)\\
      &= \frac{1}{2} \left[ \phi_{\text{odd}}\left( \frac{2}{3} \right)
      + \phi_{\text{odd}}\left( \frac{2}{3} \right)\right] + \left( \int_{0}^{1} \psi(s)\ ds
      - \int_{0}^{1} \psi (s)\ ds \right)\\
      &= \frac{1}{2} \left[ \phi \left( \frac{2}{3} \right)
      + \phi \left( \frac{2}{3} \right) \right] + 0\\
      &= \phi \left( \frac{2}{3} \right)\\
      &= \left( \frac{2}{3} \right)^{2} \left( 1 - \frac{2}{3} \right)\\
      &= \left( \frac{4}{9} \right) \left( \frac{1}{3} \right)\\
      &= \frac{4}{27}
    \end{align*}
    \newpage
    For $u(\frac{1}{4}, \frac{7}{2})$ we get that
    \begin{align*}
      u \left( \frac{1}{4}, \frac{7}{2} \right)
      &= \frac{1}{2} \left[ \phi_{\text{odd}} \left( \frac{1}{4} + \frac{7}{2} \right)
      + \phi_{\text{odd}} \left( \frac{1}{4} - \frac{7}{2} \right) \right]
      + \frac{1}{2} \int_{\frac{1}{4} - \frac{7}{2}}^{\frac{1}{4} + \frac{7}{2}}
      \psi_{\text{odd}}(s)\ ds\\
      &= \frac{1}{2} \left[ \phi_{\text{odd}} \left( \frac{15}{4} \right)
      + \phi_{\text{odd}} \left( -\frac{13}{4} \right) \right]
      + \frac{1}{2} \int_{-\frac{13}{4}}^{\frac{15}{4}} \psi_{\text{odd}}(s)\ ds\\
      &= \frac{1}{2} \left[ -\phi\left(\frac{1}{4}\right) + \phi\left(\frac{3}{4}\right) \right]
      + \frac{1}{2} \left( 3 \int_{0}^{1} \psi(s)\ ds - 3 \int_{0}^{1} \psi(s)\ ds
      + \int_{\frac{3}{4}}^{1} \psi(s)\ ds + \int_{\frac{1}{4}}^{1} -\psi(s)\ ds\right)\\
      &= \frac{1}{2} \left[ -\phi \left( \frac{1}{4} \right) + \phi \left( \frac{3}{4} \right) \right]
      + \frac{1}{2} \left( \int_{\frac{3}{4}}^{1} \psi(s)\ ds
      + \int_{1}^{\frac{1}{4}} \psi(s)\ ds\right)\\
      &= \frac{1}{2} \left[ -\phi \left( \frac{1}{4} \right)
      + \phi \left( \frac{3}{4} \right) \right]
      + \frac{1}{2} \left( \int_{\frac{3}{4}}^{\frac{1}{4}} \psi(s)\ ds\right)\\
      &= \frac{1}{2} \left[ -\phi \left( \frac{1}{4} \right)
      + \phi \left( \frac{3}{4} \right) \right]
      - \frac{1}{2} \int_{\frac{1}{4}}^{\frac{3}{4}} \psi(s)\ ds\\
      &= \frac{1}{2} \left[
      \left( \left( \frac{1}{4} \right)^{2} \left( \frac{1}{4} - 1 \right) \right)
      + \left( \left( \frac{3}{4} \right)^{2} \left( 1 - \frac{3}{4} \right) \right)
      \right]
      - \frac{1}{2} \int_{\frac{1}{4}}^{\frac{3}{4}} (1 - s)^2\ ds\\
      &= \frac{1}{2} \left[
      \left(-\frac{3}{64}\right) + \frac{9}{64}
      \right]
      - \frac{1}{2} \left(
        -\frac{1}{3}\left( 1 - s \right)^{3}
        \Bigg|_{\frac{1}{4}}^{\frac{3}{4}}
      \right)\\
      &= \frac{1}{2} \left( \frac{3}{32} \right)
      + \frac{1}{6} \left( (1 - s)^{3} \Bigg|_{\frac{1}{4}}^{\frac{3}{4}} \right)\\
      &= \frac{3}{64} + \left( -\frac{13}{192} \right)\\
      &= -\frac{1}{48}
    \end{align*}
  \end{solution}
\end{document}

